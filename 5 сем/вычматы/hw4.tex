\documentclass[12pt]{article}
\usepackage[T2A]{fontenc}
\usepackage[utf8]{inputenc}
\usepackage[russian]{babel}
\usepackage{amsmath, amssymb}
\usepackage{geometry}
\geometry{a4paper, left=20mm, right=15mm, top=20mm, bottom=20mm}

\title{Задача III.5.11.в$)$ Наилучшая среднеквадратичная линейная аппроксимация $f(x) = \ln(1+x)$ на $[0, 1]$}
\author{Хамаш Виктории Б01-301}
\date{}
\begin{document}

\maketitle

\section*{Постановка задачи}
Требуется построить наилучшую среднеквадратичную линейную аппроксимацию функции $f(x) = \ln(1+x)$ на отрезке $[0, 1]$ в виде:
\[
F(x) = u_0 + u_1x
\]
Коэффициенты $u_0$ и $u_1$ находятся методом наименьших квадратов (МНК) путём минимизации функционала:
\[
\Phi(u_0, u_1) = \int_0^1 (F(x) - f(x))^2 dx = \int_0^1 (u_0 + u_1x - \ln(1+x))^2 dx
\]

\section*{Теоретическое обоснование}
Согласно Лекции 3 (МНК для непрерывного случая), для минимизации функционала $\Phi$ необходимо решить систему линейных алгебраических уравнений (СЛАУ) относительно $u_0$ и $u_1$:
\begin{align*}
(\varphi_0, \varphi_0)u_0 + (\varphi_0, \varphi_1)u_1 &= (f, \varphi_0) \\
(\varphi_1, \varphi_0)u_0 + (\varphi_1, \varphi_1)u_1 &= (f, \varphi_1)
\end{align*}
где:
\begin{itemize}
    \item Базисные функции: $\varphi_0(x) = 1$, $\varphi_1(x) = x$
    \item Скалярное произведение: $(g, h) = \int_0^1 g(x)h(x) dx$
\end{itemize}

\section*{Вычисление элементов системы}

\subsection*{Элементы матрицы Грама}
\begin{align*}
(\varphi_0, \varphi_0) &= \int_0^1 1 \cdot 1 dx = \left[ x \right]_0^1 = 1 \\
(\varphi_0, \varphi_1) = (\varphi_1, \varphi_0) &= \int_0^1 1 \cdot x dx = \left[ \frac{x^2}{2} \right]_0^1 = \frac{1}{2} \\
(\varphi_1, \varphi_1) &= \int_0^1 x \cdot x dx = \left[ \frac{x^3}{3} \right]_0^1 = \frac{1}{3}
\end{align*}

\subsection*{Элементы правой части}
\begin{align*}
(f, \varphi_0) &= \int_0^1 \ln(1+x) \cdot 1 dx = I_0 \\
(f, \varphi_1) &= \int_0^1 \ln(1+x) \cdot x dx = I_1
\end{align*}

Вычислим интегралы $I_0$ и $I_1$.

\paragraph*{Вычисление $I_0$:}
Применим интегрирование по частям. Положим:
\[
\begin{cases}
u = \ln(1+x) \Rightarrow du = \frac{dx}{1+x} \\
dv = dx \Rightarrow v = x
\end{cases}
\]
Тогда:
\[
I_0 = \left[ x \ln(1+x) \right]_0^1 - \int_0^1 \frac{x}{1+x} dx
\]
Вычислим оставшийся интеграл:
\[
\int_0^1 \frac{x}{1+x} dx = \int_0^1 \left(1 - \frac{1}{1+x}\right) dx = \left[ x - \ln(1+x) \right]_0^1 = (1 - \ln 2) - (0 - 0) = 1 - \ln 2
\]
Подставляем:
\[
I_0 = (1 \cdot \ln 2) - (1 - \ln 2) = \ln 2 - 1 + \ln 2 = 2\ln 2 - 1
\]

\paragraph*{Вычисление $I_1$:}
Также применим интегрирование по частям. Положим:
\[
\begin{cases}
u = \ln(1+x) \Rightarrow du = \frac{dx}{1+x} \\
dv = x dx \Rightarrow v = \frac{x^2}{2}
\end{cases}
\]
Тогда:
\[
I_1 = \left[ \frac{x^2}{2} \ln(1+x) \right]_0^1 - \int_0^1 \frac{x^2}{2} \cdot \frac{1}{1+x} dx = \frac{1}{2}\ln 2 - \frac{1}{2} \int_0^1 \frac{x^2}{1+x} dx
\]
Разложим подынтегральное выражение:
\[
\frac{x^2}{1+x} = \frac{x^2 - 1 + 1}{1+x} = \frac{(x-1)(x+1)}{1+x} + \frac{1}{1+x} = (x-1) + \frac{1}{1+x}
\]
Вычислим интеграл:
\[
\int_0^1 \frac{x^2}{1+x} dx = \int_0^1 \left(x - 1 + \frac{1}{1+x}\right) dx = \left[ \frac{x^2}{2} - x + \ln(1+x) \right]_0^1 = \left(\frac{1}{2} - 1 + \ln 2\right) - (0) = \ln 2 - \frac{1}{2}
\]
Подставляем обратно в выражение для $I_1$:
\[
I_1 = \frac{1}{2}\ln 2 - \frac{1}{2} \left( \ln 2 - \frac{1}{2} \right) = \frac{1}{2}\ln 2 - \frac{1}{2}\ln 2 + \frac{1}{4} = \frac{1}{4}
\]

\section*{Решение системы уравнений}
Система принимает вид:
\begin{align}
u_0 + \frac{1}{2}u_1 &= 2\ln 2 - 1 \\
\frac{1}{2}u_0 + \frac{1}{3}u_1 &= \frac{1}{4}
\end{align}
Подставим численное значение $\ln 2 \approx 0.693147$ в уравнение (1):
\[
u_0 + 0.5u_1 = 2 \cdot 0.693147 - 1 = 1.386294 - 1 = 0.386294
\]
Умножим уравнение (2) на 2 для упрощения:
\[
u_0 + \frac{2}{3}u_1 = 0.5
\]
Теперь вычтем уравнение (1) из полученного уравнения:
\[
\left(u_0 + \frac{2}{3}u_1\right) - \left(u_0 + 0.5u_1\right) = 0.5 - 0.386294
\]
\[
\left(\frac{2}{3} - \frac{1}{2}\right)u_1 = 0.113706
\]
\[
\left(\frac{4}{6} - \frac{3}{6}\right)u_1 = 0.113706
\]
\[
\frac{1}{6}u_1 = 0.113706
\]
\[
u_1 = 0.113706 \cdot 6 = 0.682236
\]
Теперь найдём $u_0$ из уравнения (1):
\[
u_0 + 0.5 \cdot 0.682236 = 0.386294
\]
\[
u_0 + 0.341118 = 0.386294
\]
\[
u_0 = 0.386294 - 0.341118 = 0.045176
\]

\section*{Ответ}
Наилучшая среднеквадратичная линейная аппроксимация функции $f(x) = \ln(1+x)$ на отрезке $[0, 1]$ имеет вид:
\[
\boxed{F(x) \approx 0.04518 + 0.68224x}
\]


\end{document}